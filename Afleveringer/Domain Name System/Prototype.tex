\documentclass[Preamble]{subfiles}
\begin{document}

\chapter{Prototype: School DNS}
A school want to use a DNS server to filter certain internet sites from students and also have the opportunity to get a faster response from the name server. 

\section{Solution}
To achieve the school's request, a BIND server could be set up. BIND is a open source implementation of the DNS protocols and is the most used DNS server software. One of the advantages with BIND, is that it supports both Windows, Mac and Linux. BIND acts like a caching server, where it stores answers to name queries and this results in reduced time of future queries to the same server.

ANDERS : LAV TEST MED DIG -X

BIND can afterwards be configured in different ways to achieve a filter. One of the solutions is forwarding to a public DNS and another is local configuring.

\subsection{Forward to public DNS}

LASSE : TEST MED PUBLIC DNS - UDGANGSPUNKT I BILLEDE + Målinger

One solution to the school case is to forward all their requests to a public DNS, e.g. OpenDNS. This would be a simple solution, that for some servers would give a faster response. Furthermore some servers are filtering sites that can harm your computer, and thereby make it safer to use the network.

There is a lot of public DNS servers, but not all will make the respond time faster. To find  an optimal solution, Google's Test Bench (GTB) have been used. In this case GTB looked up around 4500 servers and tested them all to find the fastest server in average.

After the tests the fastest server can be written into /etc/bind/named

To verify the tests a comparison of three scenarios have been made: No forwarder, forwarder to

\subsection{Local filtering}
TODO - Skal skrives om hvordan man kan udbygge filtering ved localhost og spærre for decideret sider. 

\section{Setup BIND Server}
To install a BIND server on Linux type in "sudo apt-get install bind[9]". This will install version 9 of the BIND server software. To check if installation if succesfull type "named -v" and if it is successfull, it will show "BIND 9.8.1-P1". For testing purpose, "dnsutils" have been used - and this can be used to see Query time for the DNS lookup with the commannd "dig -x IP-Adress". 

\end{document} 