\documentclass[Main]{subfiles}
\begin{document}

\chapter{Conclusion}

\section{Conclusion}
To make the development of a distributed system easier developers can use a middleware which is  software installed on all the implicated computers. 
The middleware-software is between the application and the physical layer which makes communication between applications in different programming languages on different hardware and operating systems easy.

The characteristics of a middleware are:

\begin{itemize}
\item Adaptive
\\
The middleware makes applications adaptable to changing availability of system resources.

\item Flexibility and scalability
\\
It is easy to add or remove more computers to the network as the middleware can offer space-decoupling; the implicated computers does not know about each others existence.

\item Security
\\
Can allow applications authorized access to resources. Also all the implicated computers need to be on the same domain in order to communicate.

\item Portability
\\
Middleware hides the problems that can occur due to different hardware/software environments.

\item Connectivity and intercommunication
\\
The main functionality of middlewares is that it enables a application to share data and communicate transparently with other applications on the network.

\end{itemize}

These characteristics makes it easier to write complex applications with the aim of sharing or transferring data. 
Furthermore the developers can focus on the application and the middleware-interface instead of concerning about the link and physical layers.
\\
The most middleware are proprietary but it is possible to use open source middleware if wanted.


The implementation of a middleware can use a standard which ensure that most partners can collaborate when sharing data as they use the same interface. 
The DDS standard is using a data-centric publisher/subscribe communication paradigm where publishers and subscribers communicate using categorized messages which are not send to a specific subscriber. 
The subscribers subscribes to specific categories and can filter on topic, content or type. 
This kind of communication involves a loose coupling which enable large-scale and dynamic networks.





\section{Discussion}
Whether or not to use a middleware depends on the size and the aims of the network as the software need to be installed on all implicated computers. 
If the network only consists of few computers it can be easier for the developers not to use middleware. 
But if applications made by different developers it can be an advantage to use a middleware as the developers only have to learn about the middleware-interface and not take the possible hardware-, software-, and/or languages-differences into account.
 Furthermore middleware enables scalability of the network, but if it is certain that a small network will not expand, it might not be recommended to use a middleware. 
 Developers should also consider which kind of middleware to use as different vendors provides middleware aimed for certain purposes. 







\section{Perspectives}
The middleware technology is becoming even more important and necessary in the future because technology evolves very fast making the technical environment heterogeneous. 
Distributed systems will grow and include new hard- and software which require that older hard- and software must be able to collaborate with it  \cite{whereIsMiddleware}.
\\
Computer technology is becoming more and more ubiquitous in our daily living; e.g. awareness technologies letting you know where family-members are located, assisted living in elderly care, home-monitoring of chronical patients or use of different sensors during admissions.
\\
In health care different kind of information about the patient/citizen, their location, vital parameters etc. is useful in various records, e.g. \textit{den Elektroniske PatientJournal}, \textit{den Elektroniske BorgerJournal} or at the general practitioner.
\\
Data and information distribution in large or complex systems can benefit from the use of middleware especially were a publisher/subscriber communication paradigm is used as the middleware will take care of the transmission ensuring that all relevant receivers will get the wanted information.






\end{document} 