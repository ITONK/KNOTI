\documentclass[Main]{subfiles}
\begin{document}

\chapter{Conclusion}

\section{Conclusion}
To make the development of a distributed system easier developers can use a middleware which is a software installed on all the implicated computers. The middleware-software is between the application and the physical layer which makes communication between applications in different programming languages on different hardware and operating systems easy.

The characteristics of a middleware is:
\begin{itemize}
\item[Adaptive]\\The middleware makes applications adaptable to changing availability of system resources.
\item[Flexibility and scalability]\\It is easy to add or remove more computers to the network as the middleware can offer space-decoupling; the implicated computers does not know about each others existence.
\item[Security]\\
\item[Portability]\\Middleware hides the problems that can occur due to different hardware/software environments.
\item[Connectivity and intercommunication]\\The main functionality of middlewares is that it enables a application to share data and communicate transparently with other applications on the network.
\end{itemize}

These characteristics makes it easier to write complex applications with the aim of sharing or transferring data. Furthermore the developers can focus on the application and the middleware-interface instead of concerning about the link and physical layers.

\section{Discussion}
open middleware vs. proprietary middleware

\section{Perspectives}
What are the perspectives on the technology and your prototype? 


\end{document} 