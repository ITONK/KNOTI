\documentclass[a4paper,10pt]{report}

\input{../preamble}

%Begin document
%----------------------------------------------------------
\begin{document}

\begin{titlepage}
\begin{center}
{\LARGE \textbf{Data Distributed System}}


\vspace{4cm}
\textbf{Handed in May 13, 2013, by Team 2}\\~\\
\begin{tabular}{ll}
Rasmus Bækgaard  & 10893@iha.dk \\
Anders Kielsholm  & 10749@iha.dk \\
Lasse Hansen  & 10063@iha.dk \\
Mia Leth Sørensen & 10959@iha.dk \\
\end{tabular}
\vfill
\textbf{Electrical and Computer Engineering}\\
\textbf{Aarhus University}\\
\textbf{Finlandsgade 22, 8200 Aarhus N, Denmark}
\end{center}
\end{titlepage}

\subfile{Abstract}

\subfile{Introduction}

\subfile{Subject}

\subfile{Prototype}

\subfile{Conclusion}


%\nocite{*}
\bibliographystyle{ieeetr}
\bibliography{references}

%\end{thebibliography}

\end{document}