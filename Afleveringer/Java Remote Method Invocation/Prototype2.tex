\documentclass[Main]{subfiles}
\begin{document}


The prototype is designed with a \textbf{missing}
%\code{main} 
to run each client.
Upon starting the main there must be provided 4 parameters; the port number, the file listing all clients, a name and a task.
The four parameters will be read in the main-file to setup the client for the \textbf{missing}
%\code{RMIRegistry}
which dictates the port used, name of the client and the task to be performed.
Also a text file with all clients must be given.

Besides the program arguments the program requires VM arguments in the form of system properties, here the IP-address of it self; \textbf{missing}.
%\code{-Djava.rmi.server.hostname=10.193.2.91}

The \textbf{missing} 
%\code{main}
now creates the registry for the RMI with the port, creates a \textbf{missing}
%\code{ClientDTO}
with the port, name, and IP and checks the provided text file of clients -- to see if there's anyone to connect to and if its own information is correct.
When validated a \textbf{missing}
%\code{Client}
is created and exposed as a \textbf{missing} 
%\code{RemoteObject}
, bounded to the registry and ready for use. 
Since it does not know who is coordinating the clients it starts an election and starts running it's task.



A \textbf{Missing} 
%\code{Client}
is build on a \textbf{missing}
%\code{Bully}
-interface, which inherits from a \textbf{missing}
%\code{Remote}
-class.
When it is first setup in the \textbf{missing}
%\code{main}
it will first check to see if there is any leader or it is leader it self.
If this is not the case it will ask for the leader what to perform.
In case it can't find the leader it will start an election for a new leader.



When trying to locate a new leader with the \textbf{missing}
%\code{StartElection()}
-command it first looks at the list of clients connected. 
If there is someone higher than itself it will ask them to start an election.
If such is found the function ends since another will be leader.
However if none is found a flag is set to indicated that it is the highest coordinator, as seen on \codeTitle \ref{lst:startElection} 


\begin{lstlisting}[caption=Client: StartElection(), style=Code-Java, label=lst:startElection]
for(ClientDTO client : higherThanMe) {
	try 
	{
		Registry registry = LocateRegistry.getRegistry(client.host, client.port);
		Bully stub = (Bully) registry.lookup("Bully"); //Find one to elect
		stub.StartElection(); 
		foundHigher = true; 
	}
	...
}
\end{lstlisting}


Should none be higher that the client it declares it self for coordinator. 
First it finds everyone beneath it in the list and call \textbf{missing}
%\code{Announce(ClientDTO)}
for all as shown in \codeTitle \ref{lst:foundHigher}.


\begin{lstlisting}[caption=Client: StartElection(), style=Code-Java, label=lst:foundHigher]
if(!foundHigher)
{
	leader = clientInfo;

	for(ClientDTO client : lowerThanMe) {
		try 
		{
			Registry registry = LocateRegistry.getRegistry(client.host, client.port);
			Bully stub = (Bully) registry.lookup("Bully");
			stub.Announce(clientInfo); 
		}
		catch (ConnectException e) 
		{
			System.out.println("Can't connect to " + client.name + ". It just misses the announcement.");
		}
	}
	if(lowerThanMe.size() == 0)
	{
		System.out.println("I am last client and leader");
	}
}
\end{lstlisting}


Testing the system shows that it works as intended.
When clients higher than the coordinator comes online the election takes place, when a coordinator disconnects a new takes over and clients beneath the coordinator can come and go as they like without interrupting the whole system.
Specifying the IP-addresses is in a file all systems must have available is not idea and a solution like the DNS-system would be preferred\cite[p. 4]{OurDnsReport}.


\end{document}