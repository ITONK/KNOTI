\documentclass[Main]{subfiles}
\begin{document}

\chapter*{Abstract}
\addcontentsline{toc}{chapter}{Abstract} 

When organizing distributed systems the paradigm of distributed objects can be used. This paradigm offers the ability to build software into well-defined and somewhat independent components which enable developers to concentrate on functionality in their software independent from other developers. The features of the server-object(s) are made available to clients via method-invocations through an interface.\\RMI gives a client on one Java Virtual Machine (JVM) access to objects that runs on another JMV which allows the client to invoke methods on the object - called the Remote Object. A JVM is a program that provides a run-time environment where applications written in Java binary code (called bytecode) can be executed. The JVM contains a set of instructions which is used when interpreting Java bytecode and this enable the processor to execute a compiled Java program.


The architecture of the RMI is based on a client-server relationship with three main layers that handles the access and method-invocation between processes. To coordinate the collaboration a leader among the processes is chosen. This leader election can be made by implementing the algorithm \textit{Bully election}, where the process with the highest ID (could be the IP or MAC-address) is the leader. If the leader-process fails or is being retrieved from the network a new leader is chosen by comparing ID's.



\setcounter{tocdepth}{1}
\tableofcontents



\end{document} 